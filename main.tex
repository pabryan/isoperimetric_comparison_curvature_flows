\documentclass{amsart}

\input{StandardPaper2.tex}

\begin{document}

\title[Isoperimetric Comparison]
 {Isoperimetric Comparisons For Intrinsic Curvture Flows}

\curraddr{}
\email{}
\date{\today}

\dedicatory{}
\subjclass[2010]{}
\keywords{}

\begin{abstract}
\end{abstract}

\maketitle

\section{Introduction}
\label{sec:intro}

Let \(M\) be a smooth manifold and \(g = g_t\) a smooth one-parameter family of metrics evolving by,
\begin{equation}
\label{eq:dtg}
\partial_t g = 2 X.
\end{equation}

The Isoperimetric Profile is defined by,
\[
I_t (x) = I(x, t) = \inf\{\abs{\bdry{\Omega}}_{g_t} : \abs{\Omega}_{g_t} = x\}
\]
and the infimum ranges over all compactly contained, connected, open sets \(\Omega\) with smooth boundary. An isoperimetric domain is such a set \(\Omega\) satisfying \(I(\abs{\Omega}_{g_t}, t) = \abs{\bdry{\Omega_0}}_{g_t}\).

\section{A Differential Inequality For The Isoperimetric Profile}
\label{sec:iso_diff_ineq}

Let \(\Omega_0 \subset M\) be an isoperimetric domain at time \(t = t_0\) so that,
\[
I_{t_0} (\abs{\Omega_0}_{t_0}) = \abs{\bdry{\Omega_0}}_{t_0}.
\]

\subsection{Variation Formulae}
\label{subsec:iso_diff_ineq_variation}

Let also \(\Omega_u\) denote a smooth variation of \(\Omega_0\). That is, there exists an \(\epsilon > 0\) and a smooth map \(\phi: \Omega_0 \times (-\epsilon, \epsilon) \to M\) such that, writing \(\phi_u(\cdot) = \phi(\cdot, u)\) and \(\Omega_u = \phi_u(\Omega_0)\), we have \(\phi_0\) is the inclusion \(\Omega_0 \to M\) and for each \(u\), \(\phi_u\) is an embedding. Let us also write \(V = \phi_{\star} \partial_u\) for the variation vector field and \(\phi = g_{t_0} (V, \nu)\) for the normal component where \(\nu\) is the outer unit normal normal vector field to \(\bdry{\Omega_0}\). We assume that \(\phi \equiv 1\).

With our assumptions, the standard first variation formulae at \(u = 0\) for volume and perimeter read,
\[
\partial_u \abs{\Omega_u} = \int_{\bdry{\Omega_0}} \sigma = \abs{\bdry{\Omega_0}}, \quad \partial_u \abs{\bdry{\Omega_u}} = \int_{\bdry{\Omega_0}} H \sigma = H_0 \abs{\bdry{\Omega_0}}
\]
where \(H\) is the mean curvature of \(\bdry{\Omega_0}\) which is constant \(H \equiv H_0\) since \(\Omega_0\) is isoperimetric. The second variations of perimeter which are
\[
\partial_u^2 \abs{\Omega_u} = H_0 \abs{\bdry{\Omega_0}}
\]
and
\[
\partial_u^2 \abs{\bdry{\Omega_u}} = H_0^2 \abs{\bdry{\Omega_0}} - \int_{\bdry{\Omega_0}} \abs{A}^2 + \ric(\nu) \sigma.
\]

The time variation of \(\abs{\Omega_0}\) may be computed from \(\partial_t \mu = \tfrac{1}{2} \Tr (g^{-1} \otimes 2X)\mu\),
\[
\partial_t \abs{\Omega_0} = \int_{\Omega_0} \Tr_g X \mu.
\]

\begin{lemma}
\[
\partial_t \abs{\bdry{\Omega_u}} = \int_{\bdry{\Omega_0}} \Tr_g X - X(\nu, \nu) \sigma.
\]
\end{lemma}

\begin{proof}
We use the definition,
\[
\sigma = \intprod_{\nu} \mu = \Tr \nu \otimes \mu.
\]
Then,
\begin{equation}
\label{eq:dtsigma}
\partial_t \sigma = \Tr(\partial_t \nu \otimes \mu) + \Tr(\nu \otimes \partial_t \mu).
\end{equation}

The second term is simple,
\begin{equation}
\label{eq:dtsigma2}
\begin{split}
\Tr(\nu \otimes \partial_t \mu) &= \Tr(\nu \otimes \Tr(g^{-1} \otimes X) \mu) \\
&= \Tr(g^{-1} \otimes X) \Tr (\nu \otimes \mu) \\
&= \Tr(g^{-1} \otimes X) \sigma.
\end{split}
\end{equation}

For the first term, we must compute \(\partial_t \nu\). Note that \(\bdry{\Omega_0}\) itself is not changing, whereas the metric is changing, and hence so too is \(\nu\) which is defined via,
\begin{equation}
\label{eq:normalequation}
g_t(\nu, \nu) = 1, \quad g_t(\nu, X) = 0 \text{ for every \(X\) tangent to \(\bdry{\Omega_0}\)}.
\end{equation}

From the first defining equation in \eqref{eq:normalequation} and the evolution of the metric, \eqref{eq:dtg}
\[
0 = \partial_t (g(\nu, \nu)) = (\partial_t g) (\nu, \nu) + 2g(\partial_t \nu, \nu)
\]
so that
\[
g(\partial_t \nu, \nu) = -X(\nu, \nu).
\]
Thus we obtain,
\begin{equation}
\label{eq:dtsigma1}
\Tr(\partial_t \nu \otimes \mu) = g(\partial_t \nu, \nu) \sigma = -X(\nu, \nu) \sigma.
\end{equation}

Substitution of \eqref{eq:dtsigma1} and \eqref{eq:dtsigma2} into \eqref{eq:dtsigma} completes the proof.
\end{proof}

\begin{rem}
For the record, the second defining equation in \eqref{eq:normalequation}, the evolution of the metric, \eqref{eq:dtg} and \(\partial_t X = 0\) give
\[
g(\partial_t \nu, X) = -(\partial_t g) (\nu, X) - g(\nu, \partial_t X) = -2 X(\nu, X).
\]
\end{rem}

\subsection{Viscosity Equation}
\label{subsec:iso_diff_ineq_viscosity}

\begin{thm}
\label{thm:general_viscosity}
The isoperimetric profile is a viscosity super-solution of,
\[
\begin{split}
\partial_t I &\geq I^2 I'' +  \int_{\bdry{\Omega_0}} \left(\abs{A}^2 + \ric(\nu, \nu)\right) \sigma  - I'\int_{\Omega_0} \Tr_g X \mu + \int_{\bdry{\Omega_0}} \left(\Tr_g X - X(\nu, \nu)\right) \sigma \\
&= I^2 I'' +  (I')^2 I + \int_{\bdry{\Omega_0}} \left(Tr_g X + R_{\Omega_0} - X(\nu, \nu) - \ric(\nu)\right) \sigma \\
&\quad - \left(I'\int_{\Omega_0} \Tr_g X \mu + \int_{\bdry{\Omega_0}} R_{\bdry{\Omega_0}}\sigma \right) .
\end{split}
\]
where \(\Omega_0 = \Omega_0(x_0, t_0)\) is an isoperimetric domain at \((x_0, t_0)\).
\end{thm}

\begin{rem}
At this level of generality, we cannot say anything more. The first line of each equation depends on the speed \(X\) and it's relation to the ambient curvature along an (essentially unknown) isoperimetric domain. To go further we must specify the speed which we do later. More difficult to deal with is the second line, which also depends on the speed \(X\) but cannot be balanced against the other terms unless we can apply some (generally topological) argument to rewrite it as a boundary integral. We also address this issue below.
\end{rem}

\begin{proof}
Let \(\phi \leq_{x=x_0, t\leq t_0} I\) be smooth, lower supporting function, \(\Omega_0\) and isoperimetric domain for \((x_0, t_0)\) and \(\Omega_u\) a variation of \(\Omega_0\). Define,
\[
\Phi(u, t) = \abs{\bdry{\Omega_u}}_t - \phi(\abs{\Omega_u}_t, t).
\]
Then \(\Phi\) is smooth and since \(\abs{\Omega_u}_t \geq I(\abs{\Omega_u}_t, t) \geq \phi(\abs{\Omega_u}_t, t)\), we have \(\Phi \geq 0\) for \(u \in (-\epsilon, \epsilon)\) and \(t \leq t_0\) along with \(\Phi(0, t_0) = 0\). Note then that
\begin{equation}
\label{eq:viscosity_identities}
\abs{\Omega_0} = x_0, \quad I(x_0, t_0) = \abs{\bdry{\Omega_0}} = \phi(x_0, t_0).
\end{equation}

Therefore, at \((0, t_0)\), the first variation in \(u\) of \(\Phi\) vanishes giving,
\[
H_0 \abs{\bdry{\Omega_0}} = \partial_u \abs{\bdry{\Omega_u}}_t = \phi' \partial_u \abs{\Omega_u}_t
\]
so that
\[
\phi' = H_0.
\]
Also, at \((0, t_0)\), the time variation is non-positive, and the second spatial variation is non-negative, hence
\[
\begin{split}
0 &\geq (\partial_t - \partial_u^2) \Phi \\
&= (\partial_t - \partial_u^2) \abs{\bdry{\Omega}} - \phi'(\partial_t - \partial_u^2) \abs{\Omega_u} + \left(\partial_u \abs{\Omega_u}\right)^2 \phi'' - \partial_t \phi.
\end{split}
\]

From the variation formulae, and using \eqref{eq:viscosity_identities} we obtain
\[
(\partial_t - \partial_u^2) \abs{\bdry{\Omega}} = \int_{\bdry{\Omega_0}} \left(\Tr_g X - X(\nu, \nu) + \abs{A}^2 + \ric(\nu, \nu)\right) \sigma - (\phi')^2 \phi,
\]
\[
(\partial_t - \partial_u^2) \abs{\Omega_u} = \int_{\Omega_0} \Tr_g X \mu - \phi \phi',
\]
and
\[
\left(\partial_u \abs{\Omega_u}\right)^2 = \phi^2.
\]

Putting this together we arrive at,
\[
\begin{split}
\partial_t \phi &\geq \int_{\bdry{\Omega_0}} \left(\Tr_g X - X(\nu, \nu) + \abs{A}^2 + \ric(\nu, \nu) \right)\sigma - (\phi')^2 \phi \\
&\quad - \phi'\int_{\Omega_0} \Tr_g X \mu + \phi (\phi')^2 \\
&\quad \phi^2 \phi'' -\partial_t \phi \\
&= \phi^2 \phi'' + \int_{\bdry{\Omega_0}} \left(\abs{A}^2 + \ric(\nu, \nu)\right) \sigma - \phi'\int_{\Omega_0} \Tr_g X \mu + \int_{\bdry{\Omega_0}} \left(\Tr_g X - X(\nu, \nu)\right) \sigma.
\end{split}
\]
The result follows by the definition of viscosity super solution.

For the second form, we use the Gauss equation, which states,
\[
2 \ric(\nu) + \abs{A}^2 = R_{\Omega_0} + H^2 - R_{\bdry{\Omega_0}}.
\]
Thus we obtain,
\[
\begin{split}
\partial_t \phi - \phi^2 \phi'' &\geq \int_{\bdry{\Omega_0}} \left(\abs{A}^2 + \ric(\nu, \nu)\right) \sigma  - \phi'\int_{\Omega_0} \Tr_g X \mu + \int_{\bdry{\Omega_0}} \left(\Tr_g X - X(\nu, \nu)\right) \sigma \\
&= (\phi')^2 \phi - \phi'\int_{\Omega_0} \Tr_g X \mu \\
&\quad + \int_{\bdry{\Omega_0}} \left(-\ric(\nu) + R_{\Omega_0} - R_{\bdry{\Omega_0}} + \Tr_g X - X(\nu, \nu)\right) \sigma
\end{split}
\]
as required.
\end{proof}

By making use of the Gauss-Bonnet formula, we can improve this equation in two and three dimensions.

\begin{cor}
In two dimensions,
\[
\begin{split}
\partial_t I - I^2 I'' - I(I')^2 &\geq \int_{\bdry{\Omega_0}} \left(\Tr_g X + R_{\Omega_0} - X(\nu, \nu) - \ric(\nu)\right)\sigma \\
&\quad - I'\int_{\Omega_0} \Tr_g X \mu
\end{split}
\]
while in three dimensions,
\[
\begin{split}
\partial_t I - I^2 I'' - I(I')^2 &\geq \int_{\bdry{\Omega_0}} \left(\Tr_g X + R_{\Omega_0} - X(\nu, \nu) - \ric(\nu)\right)\sigma \\
&\quad - \left(I'\int_{\Omega_0} \Tr_g X \mu +  4\pi \chi(\bdry{\Omega_0})\right).
\end{split}
\]
\end{cor}

\begin{proof}
In two dimensions, \(\bdry{\Omega_0}\) is one dimensional, hence \(R_{\bdry{\Omega_0}} = 0\). Substitution in into the second equality in \Cref{thm:general_viscosity} gives the two dimensional result.

In three dimensions, \(\bdry{\Omega_0}\) is two dimensional, hence \(R_{\bdry{\Omega_0}} = 2K\) and the Gauss-Bonnet formula implies,
\[
\int_{\bdry{\Omega_0}} R_{\bdry{\Omega_0}} = 2\pi \chi(\bdry{\Omega_0}).
\]
where \(\chi(\bdry{\Omega_0})\) denotes the Euler characteristic and we used that \(\bdry{\bdry{\Omega_0}} = \emptyset\). The three dimensional result now also follows from substitution in into the second equality in \Cref{thm:general_viscosity}.
\end{proof}

\printbibliography

\end{document}
