\documentclass{amsart}

\input{StandardPaper2.tex}

\begin{document}

\title[Isoperimetric Comparison]
 {Isoperimetric Comparisons For Intrinsic Curvture Flows}

\curraddr{}
\email{}
\date{\today}

\dedicatory{}
\subjclass[2010]{}
\keywords{}

\begin{abstract}
\end{abstract}

\maketitle

\section{Introduction}
\label{sec:intro}

Let \(M\) be a smooth manifold and \(g = g_t\) a smooth one-parameter family of metrics evolving by,
\begin{equation}
\label{eq:dtg}
\partial_t g = 2 X.
\end{equation}

The Isoperimetric Profile is defined by,
\[
I_t (x) = I(x, t) = \inf\{\abs{\bdry{\Omega}}_{g_t} : \abs{\Omega}_{g_t} = x\}
\]
and the infimum ranges over all compactly contained, connected, open sets \(\Omega\) with smooth boundary. An isoperimetric domain is such a set \(\Omega\) satisfying \(I(\abs{\Omega}_{g_t}, t) = \abs{\bdry{\Omega_0}}_{g_t}\).

\section{A Differential Inequality For The Isoperimetric Profile}
\label{sec:iso_diff_ineq}

Let \(\Omega_0 \subset M\) be an isoperimetric domain at time \(t = t_0\) so that,
\[
I_{t_0} (\abs{\Omega_0}_{t_0}) = \abs{\bdry{\Omega_0}}_{t_0}.
\]

\subsection{Variation Formulae}
\label{subsec:iso_diff_ineq_variation}

Let also \(\Omega_u\) denote a smooth variation of \(\Omega_0\). That is, there exists an \(\epsilon > 0\) and a smooth map \(\phi: \Omega_0 \times (-\epsilon, \epsilon) \to M\) such that, writing \(\phi_u(\cdot) = \phi(\cdot, u)\) and \(\Omega_u = \phi_u(\Omega_0)\), we have \(\phi_0\) is the inclusion \(\Omega_0 \to M\) and for each \(u\), \(\phi_u\) is an embedding. Let us also write \(V = \phi_{\star} \partial_u\) for the variation vector field and \(\phi = g_{t_0} (V, \nu)\) for the normal component where \(\nu\) is the outer unit normal normal vector field to \(\bdry{\Omega_0}\). We assume that \(\phi \equiv 1\).

With our assumptions, the standard first variation formulae at \(u = 0\) for volume and perimeter read,
\[
\partial_u \abs{\Omega_u} = \int_{\bdry{\Omega_0}} \sigma = \abs{\bdry{\Omega_0}}, \quad \partial_u \abs{\bdry{\Omega_u}} = \int_{\bdry{\Omega_0}} H \sigma = H_0 \abs{\bdry{\Omega_0}}
\]
where \(H\) is the mean curvature of \(\bdry{\Omega_0}\) which is constant \(H \equiv H_0\) since \(\Omega_0\) is isoperimetric. The second variations of perimeter which are
\[
\partial_u^2 \abs{\Omega_u} = H_0 \abs{\bdry{\Omega_0}}
\]
and
\[
\partial_u^2 \abs{\bdry{\Omega_u}} = H_0^2 \abs{\bdry{\Omega_0}} - \int_{\bdry{\Omega_0}} \abs{A}^2 + \ric(\nu) \sigma.
\]

The time variation of \(\abs{\Omega_0}\) may be computed from \(\partial_t \mu = \tfrac{1}{2} \Tr (g^{-1} \otimes 2X)\mu\),
\[
\partial_t \abs{\Omega_0} = \int_{\Omega_0} \Tr_g X \mu.
\]

\begin{lemma}
\[
\partial_t \abs{\bdry{\Omega_u}} = \int_{\bdry{\Omega_0}} \Tr_g X - X(\nu, \nu) \sigma.
\]
\end{lemma}

\begin{proof}
We use the definition,
\[
\sigma = \intprod_{\nu} \mu = \Tr \nu \otimes \mu.
\]
Then,
\begin{equation}
\label{eq:dtsigma}
\partial_t \sigma = \Tr(\partial_t \nu \otimes \mu) + \Tr(\nu \otimes \partial_t \mu).
\end{equation}

The second term is simple,
\begin{equation}
\label{eq:dtsigma2}
\begin{split}
\Tr(\nu \otimes \partial_t \mu) &= \Tr(\nu \otimes \Tr(g^{-1} \otimes X) \mu) \\
&= \Tr(g^{-1} \otimes X) \Tr (\nu \otimes \mu) \\
&= \Tr(g^{-1} \otimes X) \sigma.
\end{split}
\end{equation}

For the first term, we must compute \(\partial_t \nu\). Note that \(\bdry{\Omega_0}\) itself is not changing, whereas the metric is changing, and hence so too is \(\nu\) which is defined via,
\begin{equation}
\label{eq:normalequation}
g_t(\nu, \nu) = 1, \quad g_t(\nu, X) = 0 \text{ for every \(X\) tangent to \(\bdry{\Omega_0}\)}.
\end{equation}

From the first defining equation in \eqref{eq:normalequation} and the evolution of the metric, \eqref{eq:dtg}
\[
0 = \partial_t (g(\nu, \nu)) = (\partial_t g) (\nu, \nu) + 2g(\partial_t \nu, \nu)
\]
so that
\[
g(\partial_t \nu, \nu) = -X(\nu, \nu).
\]
Thus we obtain,
\begin{equation}
\label{eq:dtsigma1}
\Tr(\partial_t \nu \otimes \mu) = g(\partial_t \nu, \nu) \sigma = -X(\nu, \nu) \sigma.
\end{equation}

Substitution of \eqref{eq:dtsigma1} and \eqref{eq:dtsigma2} into \eqref{eq:dtsigma} completes the proof.
\end{proof}

\begin{rem}
For the record, the second defining equation in \eqref{eq:normalequation}, the evolution of the metric, \eqref{eq:dtg} and \(\partial_t X = 0\) give
\[
g(\partial_t \nu, X) = -(\partial_t g) (\nu, X) - g(\nu, \partial_t X) = -2 X(\nu, X).
\]
\end{rem}

\printbibliography

\end{document}
