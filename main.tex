\documentclass{amsart}

%% Packages

%\usepackage{etoolbox}
%\makeatletter
%\let\ams@starttoc\@starttoc
%\makeatother
%\makeatletter
%\let\@starttoc\ams@starttoc
%\patchcmd{\@starttoc}{\makeatletter}{\makeatletter\parskip\z@}{}{}
%\makeatother

%\usepackage[parfill]{parskip}

\usepackage[colorlinks=true,linkcolor=blue,citecolor=blue,urlcolor=blue]{hyperref}
\usepackage{bookmark}
\usepackage{amsthm,thmtools,amssymb,amsmath,amscd}

\usepackage[bibstyle=alphabetic,citestyle=alphabetic,backend=bibtex]{biblatex}
% Biblatex bug
\makeatletter
\def\blx@maxline{77}
\makeatother
\bibliography{Bibliography}

\usepackage{fancyhdr}
\usepackage{esint}

\usepackage{enumerate}

\usepackage{pictexwd,dcpic}

\usepackage{graphicx}

%% Paper specific macros
\DeclareMathOperator{\speed}{h}


%% Theorems

\declaretheorem[name=Theorem,numberwithin=section]{thm}
\declaretheorem[name=Remark,style=remark,sibling=thm]{rem}
\declaretheorem[name=Lemma,sibling=thm]{lemma}
\declaretheorem[name=Proposition,sibling=thm]{prop}
\declaretheorem[name=Definition,style=definition,sibling=thm]{defn}
\declaretheorem[name=Corollary,sibling=thm]{cor}
\declaretheorem[name=Assumption,style=remark,sibling=thm]{ass}
\declaretheorem[name=Example,style=remark,sibling=thm]{example}


\numberwithin{equation}{section}

\usepackage{cleveref}
\crefname{lemma}{Lemma}{Lemmata}
\crefname{prop}{Proposition}{Propositions}
\crefname{thm}{Theorem}{Theorems}
\crefname{cor}{Corollary}{Corollaries}
\crefname{defn}{Definition}{Definitions}
\crefname{example}{Example}{Examples}
\crefname{rem}{Remark}{Remarks}
\crefname{ass}{Assumption}{Assumptions}
\crefname{not}{Notation}{Notation}

%Symbols
\renewcommand{\~}{\tilde}
\renewcommand{\-}{\bar}
\newcommand{\bs}{\backslash}
\newcommand{\cn}{\colon}
\newcommand{\sub}{\subset}

\newcommand{\N}{\mathbb{N}}
\newcommand{\R}{\mathbb{R}}
\newcommand{\Z}{\mathbb{Z}}
\renewcommand{\S}{\mathbb{S}}
\renewcommand{\H}{\mathbb{H}}
\newcommand{\C}{\mathbb{C}}
\newcommand{\K}{\mathbb{K}}
\newcommand{\Di}{\mathbb{D}}
\newcommand{\B}{\mathbb{B}}
\newcommand{\8}{\infty}

%Greek letters
\renewcommand{\a}{\alpha}
\renewcommand{\b}{\beta}
\newcommand{\g}{\gamma}
\renewcommand{\d}{\delta}
\newcommand{\e}{\epsilon}
\renewcommand{\k}{\kappa}
\renewcommand{\l}{\lambda}
\renewcommand{\o}{\omega}
\renewcommand{\t}{\theta}
\newcommand{\s}{\sigma}
\newcommand{\p}{\varphi}
\newcommand{\z}{\zeta}
\newcommand{\vt}{\vartheta}
\renewcommand{\O}{\Omega}
\newcommand{\D}{\Delta}
\newcommand{\G}{\Gamma}
\newcommand{\T}{\Theta}
\renewcommand{\L}{\Lambda}

%Mathcal Letters
\newcommand{\cL}{\mathcal{L}}
\newcommand{\cT}{\mathcal{T}}
\newcommand{\cA}{\mathcal{A}}
\newcommand{\cW}{\mathcal{W}}

%Mathematical operators
\newcommand{\INT}{\int_{\O}}
\newcommand{\DINT}{\int_{\d\O}}
\newcommand{\Int}{\int_{-\infty}^{\infty}}
\newcommand{\del}{\partial}

\newcommand{\inpr}[2]{\left\langle #1,#2 \right\rangle}
\newcommand{\fr}[2]{\frac{#1}{#2}}
\newcommand{\x}{\times}
\newcommand{\abs}[1]{\left|{#1}\right|}
\newcommand{\bdry}[1]{\partial {#1}}
\DeclareMathOperator{\Tr}{Tr}

\DeclareMathOperator{\intprod}{\iota}
\DeclareMathOperator{\dive}{div}
\DeclareMathOperator{\id}{id}
\DeclareMathOperator{\pr}{pr}
\DeclareMathOperator{\Diff}{Diff}
\DeclareMathOperator{\supp}{supp}
\DeclareMathOperator{\graph}{graph}
\DeclareMathOperator{\osc}{osc}
\DeclareMathOperator{\const}{const}
\DeclareMathOperator{\dist}{dist}
\DeclareMathOperator{\loc}{loc}
\DeclareMathOperator{\grad}{grad}
\DeclareMathOperator{\ric}{Ric}
\DeclareMathOperator{\Rm}{Rm}
\DeclareMathOperator{\weingarten}{\mathcal{W}}
\DeclareMathOperator{\inj}{inj}
\DeclareMathOperator{\Sc}{R}
\DeclareMathOperator{\sff}{A}

%Environments
\newcommand{\Theo}[3]{\begin{#1}\label{#2} #3 \end{#1}}
\newcommand{\pf}[1]{\begin{proof} #1 \end{proof}}
\newcommand{\eq}[1]{\begin{equation}\begin{alignedat}{2} #1 \end{alignedat}\end{equation}}
\newcommand{\IntEq}[4]{#1&#2#3	 &\quad &\text{in}~#4,}
\newcommand{\BEq}[4]{#1&#2#3	 &\quad &\text{on}~#4}
\newcommand{\br}[1]{\left(#1\right)}

%Logical symbols
\newcommand{\Ra}{\Rightarrow}
\newcommand{\ra}{\rightarrow}
\newcommand{\hra}{\hookrightarrow}
\newcommand{\mt}{\mapsto}

%Fonts
\newcommand{\mc}{\mathcal}
\renewcommand{\it}{\textit}
\newcommand{\mrm}{\mathrm}

%Spacing
\newcommand{\hp}{\hphantom}


%\parindent 0 pt

\protected\def\ignorethis#1\endignorethis{}
\let\endignorethis\relax
\def\TOCstop{\addtocontents{toc}{\ignorethis}}
\def\TOCstart{\addtocontents{toc}{\endignorethis}}


\begin{document}

\title[Isoperimetric Comparison]
 {Isoperimetric Comparisons For Intrinsic Curvature Flows}

\curraddr{}
\email{}
\date{\today}

\dedicatory{}
\subjclass[2010]{}
\keywords{}

\begin{abstract}
\end{abstract}

\maketitle

\section{Introduction}
\label{sec:intro}

Let \(M\) be a smooth manifold and \(g = g_t\) a smooth one-parameter family of metrics evolving by,
\begin{equation}
\label{eq:dtg}
\partial_t g = 2 \speed.
\end{equation}

The Isoperimetric Profile is defined by,
\[
I_t (x) = I(x, t) = \inf\{\abs{\bdry{\Omega}}_{g_t} : \abs{\Omega}_{g_t} = x\}
\]
and the infimum ranges over all compactly contained, connected, open sets \(\Omega\) with smooth boundary. An isoperimetric domain is such a set \(\Omega\) satisfying \(I(\abs{\Omega}_{g_t}, t) = \abs{\bdry{\Omega_0}}_{g_t}\).

\section{A Differential Inequality For The Isoperimetric Profile}
\label{sec:iso_diff_ineq}

\subsection{Variation Formulae}
\label{subsec:iso_diff_ineq_variation}

Our approach here is variational. Let \(\Omega_0 \subset M\) be an isoperimetric domain at time \(t = t_0\) so that,
\[
I_{t_0} (\abs{\Omega_0}_{t_0}) = \abs{\bdry{\Omega_0}}_{t_0}.
\]
Let also \(\Omega_u\) denote a smooth variation of \(\Omega_0\). That is, there exists an \(\epsilon > 0\) and a smooth map \(\phi: \Omega_0 \times (-\epsilon, \epsilon) \to M\) such that, writing \(\phi_u(\cdot) = \phi(\cdot, u)\) and \(\Omega_u = \phi_u(\Omega_0)\), we have \(\phi_0\) is the inclusion \(\Omega_0 \to M\) and for each \(u\), \(\phi_u\) is an embedding. Let \(\nu\) be the outer unit normal normal vector field to \(\bdry{\Omega_0}\). We consider unit speed, normal variations: \(\phi_{\ast} \partial_u = \nu\) along \(\bdry{\Omega_0}\) which always exist for sufficiently small \(\epsilon\) by working in a tubular neighbourhood of \(\bdry{\Omega_0}\).

\begin{lemma}
\label{lem:spatial_variation}
The first variation of volume and perimeter at \(u = 0\) are
\[
\partial_u \abs{\Omega_u} = \abs{\bdry{\Omega_0}}, \quad \partial_u \abs{\bdry{\Omega_u}} = H_0 \abs{\bdry{\Omega_0}}
\]
where \(H_0\) is the mean curvature of \(\bdry{\Omega_0}\) which is constant since is isoperimetric. The second variations are
\[
\partial_u^2 \abs{\Omega_u} = H_0 \abs{\bdry{\Omega_0}}
\]
and
\[
\partial_u^2 \abs{\bdry{\Omega_u}} = H_0^2 \abs{\bdry{\Omega_0}} - \int_{\bdry{\Omega_0}} \left(\abs{A}^2 + \ric(\nu)\right) \sigma.
\]
\end{lemma}

\begin{proof}
See \cite[Chapter 1]{Li:/2012}.
\end{proof}

\begin{lemma}
\label{lem:time_variation}
\[
\partial_t \abs{\Omega_0} = \int_{\Omega_0} \Tr_g \speed \mu,
\]
and
\[
\partial_t \abs{\bdry{\Omega_u}} = \int_{\bdry{\Omega_0}} \Tr_g \speed - \speed(\nu, \nu) \sigma.
\]
\end{lemma}

\begin{proof}
The time variation of \(\abs{\Omega_0}\) may be computed from \(\partial_t \mu = \tfrac{1}{2} \Tr (g^{-1} \otimes 2\speed)\mu\).

For the time variation of \(\bdry{\Omega_0}\) we use the definition,
\[
\sigma = \intprod_{\nu} \mu = \Tr \nu \otimes \mu.
\]
Then,
\begin{equation}
\label{eq:dtsigma}
\partial_t \sigma = \Tr(\partial_t \nu \otimes \mu) + \Tr(\nu \otimes \partial_t \mu).
\end{equation}

The second term is simple,
\begin{equation}
\label{eq:dtsigma2}
\begin{split}
\Tr(\nu \otimes \partial_t \mu) &= \Tr(\nu \otimes \Tr(g^{-1} \otimes \speed) \mu) \\
&= \Tr(g^{-1} \otimes \speed) \Tr (\nu \otimes \mu) \\
&= \Tr(g^{-1} \otimes \speed) \sigma.
\end{split}
\end{equation}

For the first term, we must compute \(\partial_t \nu\). Note that \(\bdry{\Omega_0}\) itself is not changing, whereas the metric is changing, and hence so too is \(\nu\) which is defined via,
\begin{equation}
\label{eq:normalequation}
g_t(\nu, \nu) = 1, \quad g_t(\nu, X) = 0 \text{ for every \(X\) tangent to \(\bdry{\Omega_0}\)}.
\end{equation}

From the first defining equation in \eqref{eq:normalequation} and the evolution of the metric, \eqref{eq:dtg}
\[
0 = \partial_t (g(\nu, \nu)) = (\partial_t g) (\nu, \nu) + 2g(\partial_t \nu, \nu)
\]
so that
\[
g(\partial_t \nu, \nu) = -\speed(\nu, \nu).
\]
Thus we obtain,
\begin{equation}
\label{eq:dtsigma1}
\Tr(\partial_t \nu \otimes \mu) = g(\partial_t \nu, \nu) \sigma = -\speed(\nu, \nu) \sigma.
\end{equation}

Substitution of \eqref{eq:dtsigma1} and \eqref{eq:dtsigma2} into \eqref{eq:dtsigma} completes the proof.
\end{proof}

\begin{rem}
For the record, the second defining equation in \eqref{eq:normalequation}, the evolution of the metric, \eqref{eq:dtg} and \(\partial_t X = 0\) give
\[
g(\partial_t \nu, X) = -(\partial_t g) (\nu, X) - g(\nu, \partial_t X) = -2 \speed(\nu, X)
\]
which along with \(g(\partial_t \nu, \nu) = - \speed(\nu, \nu)\) completely determines \(\partial_t \nu\).
\end{rem}

\subsection{Viscosity Equation}
\label{subsec:iso_diff_ineq_viscosity}

\begin{thm}
\label{thm:general_viscosity}
The isoperimetric profile is a viscosity super-solution of,
\[
\begin{split}
\partial_t I &\geq I^2 I'' +  \int_{\bdry{\Omega_0}} \left(\abs{A}^2 + \ric(\nu, \nu)\right) \sigma  - I'\int_{\Omega_0} \Tr_g \speed \mu + \int_{\bdry{\Omega_0}} \left(\Tr_g \speed - \speed(\nu, \nu)\right) \sigma \\
&= I^2 I'' +  (I')^2 I + \int_{\bdry{\Omega_0}} \left(Tr_g \speed + R_{\Omega_0} - \speed(\nu, \nu) - \ric(\nu)\right) \sigma \\
&\quad - \left(I'\int_{\Omega_0} \Tr_g \speed \mu + \int_{\bdry{\Omega_0}} R_{\bdry{\Omega_0}}\sigma \right) .
\end{split}
\]
where \(\Omega_0 = \Omega_0(x_0, t_0)\) is an isoperimetric domain at \((x_0, t_0)\).
\end{thm}

\begin{rem}
At this level of generality, we cannot say anything more. The first line of each equation depends on the speed \(X\) and it's relation to the ambient curvature along an (essentially unknown) isoperimetric domain. To go further, we must specify the speed which we do later. More difficult to deal with is the second line, which also depends on the speed \(\speed\) but cannot be balanced against the other terms unless we can apply some (generally topological) argument to rewrite it as a boundary integral. We also address this issue below.
\end{rem}

\begin{proof}
Let \(\phi \leq_{x=x_0, t\leq t_0} I\) be smooth, lower supporting function, \(\Omega_0\) and isoperimetric domain for \((x_0, t_0)\) and \(\Omega_u\) a variation of \(\Omega_0\). Define,
\[
\Phi(u, t) = \abs{\bdry{\Omega_u}}_t - \phi(\abs{\Omega_u}_t, t).
\]
Then \(\Phi\) is smooth and since \(\abs{\Omega_u}_t \geq I(\abs{\Omega_u}_t, t) \geq \phi(\abs{\Omega_u}_t, t)\), we have \(\Phi \geq 0\) for \(u \in (-\epsilon, \epsilon)\) and \(t \leq t_0\) along with \(\Phi(0, t_0) = 0\). Throughout this proof we tacitly make use of the variation formulae given in \Cref{lem:spatial_variation} and \Cref{lem:time_variation}.

At \((0, t_0)\), the first variation in \(u\) of \(\Phi\) vanishes giving,
\[
H_0 \abs{\bdry{\Omega_0}} = \partial_u \abs{\bdry{\Omega_u}} = \phi' \partial_u \abs{\Omega_u} = \phi' \abs{\bdry{\Omega_0}},
\]
so that \(\phi' = H_0\). We thus have the identities,
\begin{equation}
\label{eq:viscosity_identities}
\abs{\Omega_0} = x_0, \quad I(x_0, t_0) = \abs{\bdry{\Omega_0}} = \phi(x_0, t_0), \quad H_0 = \phi'(x_0, t_0).
\end{equation}

Also, at \((0, t_0)\), the time variation is non-positive, and the second spatial variation is non-negative, hence
\[
\begin{split}
0 &\geq (\partial_t - \partial_u^2) \Phi \\
&= (\partial_t - \partial_u^2) \abs{\bdry{\Omega}} - \phi'(\partial_t - \partial_u^2) \abs{\Omega_u} + \left(\partial_u \abs{\Omega_u}\right)^2 \phi'' - \partial_t \phi.
\end{split}
\]

From the variation formulae, and using \eqref{eq:viscosity_identities} we obtain
\begin{align*}
(\partial_t - \partial_u^2) \abs{\bdry{\Omega}} &= \int_{\bdry{\Omega_0}} \left(\Tr_g \speed - \speed(\nu, \nu) + \abs{A}^2 + \ric(\nu, \nu)\right) \sigma - (\phi')^2 \phi, \\
(\partial_t - \partial_u^2) \abs{\Omega_u} &= \int_{\Omega_0} \Tr_g \speed \mu - \phi \phi', \\
\intertext{and}
\left(\partial_u \abs{\Omega_u}\right)^2 &= \phi^2.
\end{align*}

Putting this together we arrive at,
\[
\begin{split}
\partial_t \phi &\geq \int_{\bdry{\Omega_0}} \left(\Tr_g \speed - \speed(\nu, \nu) + \abs{A}^2 + \ric(\nu, \nu) \right)\sigma - (\phi')^2 \phi \\
&\quad - \phi'\int_{\Omega_0} \Tr_g \speed \mu + \phi (\phi')^2 \\
&\quad \phi^2 \phi'' -\partial_t \phi \\
&= \phi^2 \phi'' + \int_{\bdry{\Omega_0}} \left(\abs{A}^2 + \ric(\nu, \nu)\right) \sigma - \phi'\int_{\Omega_0} \Tr_g \speed \mu + \int_{\bdry{\Omega_0}} \left(\Tr_g \speed - \speed(\nu, \nu)\right) \sigma.
\end{split}
\]
The inequality in the theorem follows by the definition of viscosity super solution.

For the equality, we use the Gauss equation, which states,
\[
2 \ric(\nu) + \abs{A}^2 = R_{\Omega_0} + H^2 - R_{\bdry{\Omega_0}}.
\]
Thus we obtain,
\[
\begin{split}
\partial_t \phi - \phi^2 \phi'' &\geq \int_{\bdry{\Omega_0}} \left(\abs{A}^2 + \ric(\nu, \nu)\right) \sigma  - \phi'\int_{\Omega_0} \Tr_g \speed \mu + \int_{\bdry{\Omega_0}} \left(\Tr_g \speed - \speed(\nu, \nu)\right) \sigma \\
&= (\phi')^2 \phi - \phi'\int_{\Omega_0} \Tr_g \speed \mu \\
&\quad + \int_{\bdry{\Omega_0}} \left(-\ric(\nu) + R_{\Omega_0} - R_{\bdry{\Omega_0}} + \Tr_g \speed - \speed(\nu, \nu)\right) \sigma
\end{split}
\]
as required.
\end{proof}

We can improve the theorem somewhat in two and three dimensions.

\begin{cor}
\label{cor:low_dim_general_viscosity}
In two dimensions,
\[
\begin{split}
\partial_t I - I^2 I'' - I(I')^2 &\geq \int_{\bdry{\Omega_0}} \left(\Tr_g X + R_{\Omega_0} - \speed(\nu, \nu) - \ric(\nu)\right)\sigma \\
&\quad - I'\int_{\Omega_0} \Tr_g \speed \mu
\end{split}
\]
while in three dimensions,
\[
\begin{split}
\partial_t I - I^2 I'' - I(I')^2 &\geq \int_{\bdry{\Omega_0}} \left(\Tr_g \speed + R_{\Omega_0} - \speed(\nu, \nu) - \ric(\nu)\right)\sigma \\
&\quad - \left(I'\int_{\Omega_0} \Tr_g \speed \mu +  4\pi \chi(\bdry{\Omega_0})\right).
\end{split}
\]
\end{cor}

\begin{proof}
In two dimensions, \(\bdry{\Omega_0}\) is one dimensional, hence \(R_{\bdry{\Omega_0}} = 0\). Substitution in into the second equality in \Cref{thm:general_viscosity} gives the two dimensional result.

In three dimensions, \(\bdry{\Omega_0}\) is two dimensional, hence \(R_{\bdry{\Omega_0}} = 2K\) and the Gauss-Bonnet formula implies,
\[
\int_{\bdry{\Omega_0}} R_{\bdry{\Omega_0}} = 4\pi \chi(\bdry{\Omega_0}).
\]
where \(\chi(\bdry{\Omega_0})\) denotes the Euler characteristic and we used that \(\bdry{\bdry{\Omega_0}} = \emptyset\). The three dimensional result now also follows from substitution in into the second equality in \Cref{thm:general_viscosity}.
\end{proof}

\section{Examples of Flows}
\label{sec:flows}

\subsection{The Ricci Flow}
\label{subsec:flows_ricci}

The Ricci flow is the case where $\speed = -\ric$, and so $\Tr_g \speed = -R$. The viscosity equation in \Cref{thm:general_viscosity} looks quite appealing in this case:
\begin{equation}
\label{eq:viscosity_ricciflow}
\partial_t I \geq I^2I'' + I(I')^2 + \left(I'\int_{\Omega_0} R \mu - \int_{\bdry{\Omega_0}} R_{\bdry{\Omega_0}}\sigma \right).
\end{equation}

Using \Cref{cor:low_dim_general_viscosity}, in two dimensions this becomes,
\[
\partial_t I \geq I^2I'' + I(I')^2 + I'\int_{\Omega_0} R \mu.
\]
After an application of the Gauss-Bonnet theorem, the last term becomes,
\[
I'\int_{\Omega_0} R \mu = 4\pi\chi(\Omega_0)I' - 2\int_{\bdry{\Omega_0}} H \sigma = 4\pi\chi(\Omega_0)I'- 2 I (I')^2
\]
and so we get,
\begin{equation}
\label{eq:2d_viscosity_ricciflow}
\begin{split}
\partial_t I &\geq I^2I'' - I(I')^2 - 4\pi\chi(\Omega_0)I' \\
&= \frac{1}{I^3} \Delta \ln I - 4\pi\chi(\Omega_0)I'.
\end{split}
\end{equation}
Thus we arrive at a rather more usable differential inequality with the only unknown term coming from the topology of isoperimetric domains, which depends on \(x\). Rather complete results for the closed, two-dimensional case are described in \cite{Bryan:/2016,AndrewsBryan:/2010}.

Using \Cref{cor:low_dim_general_viscosity}, in three dimensions we have
\begin{equation}
\label{eq:3d_viscosity_ricciflow}
\partial_t I \geq I^2I'' + I(I')^2 + I'\int_{\Omega_0} R \mu - 4\pi\chi(\bdry{\Omega_0}).
\end{equation}

Compared to the two-dimensional case, the second to last term presents some difficulties since it cannot (in general) be written purely in terms of topological invariants. On the other hand, it is the Einstien-Hilbert functional on an isoperimetric domain and thus has nice properties, especially in relation to the Ricci flow that we may be able to exploit.

\subsection{The Conformal, Coupled Ricci Flow }
\label{subsec:flows_ricci_coupled}

Here we take \(\speed = -\ric + \varphi g\) where \(\varphi \in C^{\infty}(M)\) is a smooth function. One may ask more generally to add a symmetric two form in place of the conformal two form \(\varphi g\). This becomes rather more complicated, so we focus on the conformal case for the time being. The conformal factor \(\varphi\) evolves by a flow coupled to the metric flow,
\[
\partial_t \varphi = v(g, \varphi)
\]
where \(v\) is a scalar function depending on the metric, the conformal factor and possibly other tensors.

Particular examples are,
\begin{align*}
\varphi &= \frac{1}{n} \bar{R} = \frac{1}{n\abs{M}} \int_M R \mu & \text{Normalised Ricci flow} \\
\varphi &= \frac{1}{n} \abs{F}^2 & \text{Ricci Yang-Mills flow \cite{Streets:/2010}}
\end{align*}

For the Ricci Yang-Mills flow, \(A\) is a connection on a \(U(1)\) bundle with curvature 2-form \(F\). If the \(U(1)\) bundle is trivial (vanishing first Chern class), then one may identify \(F\) with a two-form on \(M\) by pre-composing with a global section of the bundle. In two dimensions, any such \(F\) satisfies \(\Tr_g F \otimes F = \tfrac{1}{2} \abs{F}^2 g\) where the trace contracts one slot of \(g\) with the second slot of the first \(F\) factor and the other slot of \(g\) with the second slot of the second \(F\) factor. \(A\) is required to evolve by
\[
\partial_t A = -d^{\ast} F,
\]
which then induces the evolution
\[
\partial_t \varphi = .
\]

For the normalised Ricci flow, the conformal factor is already just curvature, depending only on the metric, and hence we don't need to think of this as a coupled flow.

Since \Cref{thm:general_viscosity} is linear in the speed \(\speed\), we simply add the conformal terms \(n \varphi\) coming from \(\Tr_g \speed\) and \(\varphi\) coming from \(\speed(\nu, \nu)\) to the Ricci flow viscosity equation \eqref{eq:viscosity_ricciflow} to obtain
\begin{equation}
\label{eq:viscosity_conformalcoupledricciflow}
\begin{split}
\partial_t I &\geq I^2I'' + I(I')^2 + \left(I'\int_{\Omega_0} R \mu - \int_{\bdry{\Omega_0}} R_{\bdry{\Omega_0}}\sigma \right) \\
&+ \int_{\bdry{\Omega_0}} (n-1) \varphi\sigma - n I' \int_{\Omega_0} \varphi \mu.
\end{split}
\end{equation}

In two dimensions, equation \eqref{eq:2d_viscosity_ricciflow} becomes

\begin{equation}
\label{eq:2d_viscosity_conformalcoupledricciflow}
\begin{split}
\partial_t I &\geq \frac{1}{I^3} \Delta \ln I - 4\pi\chi(\Omega_0)I' \\
&+ \int_{\bdry{\Omega_0}} (n-1) \varphi\sigma - n I' \int_{\Omega_0} \varphi \mu.
\end{split}
\end{equation}

In three dimensions, equation \eqref{eq:3d_viscosity_ricciflow} becomes
\begin{equation}
\label{eq:3d_viscosity_conformalcoupledricciflow}
\begin{split}
\partial_t I &\geq I^2I'' + I(I')^2 + I'\int_{\Omega_0} R \mu - 4\pi\chi(\bdry{\Omega_0}) \\
&+ \int_{\bdry{\Omega_0}} (n-1) \varphi\sigma - n I' \int_{\Omega_0} \varphi \mu.
\end{split}
\end{equation}

\subsection{The Yamabe Flow}
\label{subsec:flows_yamabe}

The Yamabe flow is the case \(\speed = -\tfrac{1}{n}(R - \bar{R})g\) where \(\bar{R} = \tfrac{1}{\abs{M}} \int_M R \mu\) is the average scalar curvature. Then \(\Tr_g \speed = -(R - \bar{R})\) and \(\speed(\nu, \nu) = -\tfrac{1}{n}(R - \bar{R})\). The viscosity equation in \Cref{thm:general_viscosity} is not quite as nice as the Ricci flow,
\[
\begin{split}
\partial_t I &\geq I^2 I'' +  (I')^2 I + \int_{\bdry{\Omega_0}} \left(-(R - \bar{R}) + R + \frac{1}{n}(R - \bar{R}) - \ric(\nu)\right) \sigma \\
&\quad + \left(I'\int_{\Omega_0} (R - \bar{R}) \mu - \int_{\bdry{\Omega_0}} R_{\bdry{\Omega_0}}\sigma \right) \\
&= I^2 I'' +  (I')^2 I + \left(1 - \frac{1}{n}\right) \bar{R} I - \bar{R}xI' - \int_{\bdry{\Omega_0}} \mathring{\ric} (\nu) \sigma \\
&\quad + I' \int_{\Omega_0} R \mu - \int_{\bdry{\Omega_0}} R_{\bdry{\Omega_0}}\sigma
\end{split}
\]
where \(\mathring{\ric} = \tfrac{1}{n}Rg - \ric\) is the traceless Ricci tensor.

In two dimensions (\(n=2\)), life is good. Since \(\bar{R} = 4\pi \chi(M)\) and \(\mathring{\ric} = 0\), using \Cref{cor:low_dim_general_viscosity} again produces,
\[
\partial_t I \geq I^2 I'' -  (I')^2 I - 4\pi\left[\chi(M)x - \chi(\Omega_0)\right]I' + 2\pi\chi(M) I.
\]
Note that this equation is the same as the normalised Ricci flow since the Yamabe flow is precisely the normalised Ricci flow in two dimensions.

In three dimensions, we only get a slight improvement,
\[
\begin{split}
\partial_t I &\geq I^2 I'' +  (I')^2 I + \left(1 - \frac{1}{n}\right) \bar{R} I - \bar{R}xI' - \int_{\bdry{\Omega_0}} \mathring{\ric} (\nu) \sigma \\
&\quad + I' \int_{\Omega_0} R \mu - 4\pi\chi(\bdry{\Omega_0}).
\end{split}
\]


\subsection{The Coupled Yamabe Flow }
\label{subsec:flows_yamabe_coupled}

Here we take \(\speed = -(R - \bar{R}) + \varphi g\) where \(\varphi \in C^{\infty}(M)\) is a smooth function. The Yamabe flow is designed to flow metric within a fixed conformal class, so it makes perfect sense to only add a conformal factor to the evolution, and the adjective ``conformal'' becomes superfluous in this case.

Particular examples are,
\begin{align*}
\varphi &= \frac{1}{n} \bar{R} - \operatorname{div} V & \text{Yamabe flow with vector metric torsion \cite{2016arXiv160609121B}} \\
\varphi &= \frac{1}{n} \abs{F}^2 & \text{Yamabe Yang-Mills flow}
\end{align*}

\subsubsection*{Yamabe flow with vector metric torsion}

Here \(V\) is a vector field generating a metric compatible connection with torsion. That is, let \(\nabla\) denote the Levi-Civita connection and define the connection \(^{V}\nabla\) to be the unique metric compatible connection with torsion tensor,
\begin{equation}
\label{eq:vector_torsion}
^{V}\nabla_X Y - ^{V}\nabla_Y X - [X, Y] = T_V(X, Y) = g(V, X) Y - g(V, Y) X.
\end{equation}
Namely, \(^{V}\nabla\) is defined by the Koszul Formula,
\begin{equation}
\label{eq:koszul}
g(^{V}\nabla_X Y, Z) = \frac{1}{2} \operatorname{Alt} \Bigl\{X[g(Y, Z)] - g(T_V(X, Z) + [X, Z], Y)\Bigr\}
\end{equation}
where $\operatorname{Alt}$ denotes the alternating sum over cyclic permutations of $X, Y, Z$. In general, for the torsion we may use any \((2, 1)\) tensor \(T\), skew-symmetric in \(X, Y\): \(T(X, Y) = -T(Y, X)\) but here we restrict to \emph{vector torsion} of the form \(T_V\).

By the Leibniz rule, and tensorality of \(\nabla Y\) and \(^{V}\nabla Y\),
\[
A_V(X, Y) = ^{V}\nabla_X Y - \nabla_X Y
\]
is tensorial. By the Koszul formula \eqref{eq:koszul} and the Koszul formula for the Levi-Civita connection \(\nabla\) (for which \(V \equiv 0\)) we have
\[
\begin{split}
2 g(A_V(X, Y), Z) &= 2 g(^{V}\nabla_X Y - \nabla_X Y, Z) = -\operatorname{Alt} g(T_V (X, Z), Y) \\
&= -g(g(V, X) Z + g(V, Z) X, Y) \\
&+ g(g(V, Z) Y - g(V, Y) Z, X) \\
&- g(g(V, Y) X + g(V, X) Y, Z) \\
&= g(g(X, Y)V - g(V, Y)X, Z)
\end{split}
\]
Thus we have $A_V(X, Y) = g(X, Y) V - g(V, Y) X$ and hence
\begin{equation}
\label{eq:vector_connection}
^{V}\nabla_X Y = \nabla_X Y + g(V, Y) X - g(X, Y) V
\end{equation}
provides an alternative, explicit expression to the Koszul formula \eqref{eq:koszul} for the unique, metric compatible connection with torsion \(T_V\).

Note that metric compatibility of $^{V}\nabla$ is equivalent to the skew-symmetry,
\[
g(A_V(X, Y), Z) = - g(A_V(X, Z), Y).
\]
Then, given such \(A_V\), the connection \(^{V}\nabla_X Y = \nabla_X Y + A_V(X, Y)\) is a metric compatible connection with torsion \(T_V\) given by equation \eqref{eq:vector_torsion}.

\begin{rem}
On a two dimensional, closed manifold such skew-symmetry is enough to ensure the existence of a vector field \(V\) such that \(A(X, Y) = g(V, X)Y\) \cite[Theorem 3.1]{MR712664}. In other words, on a surface, all torsion vectors are vector torsion vectors. For higher dimensions, here we consider only vector torsion.
\end{rem}

In any case, the vector field \(V\) is required to evolve by
\[
\partial_t V = -(R - \bar{R}) V.
\]

\subsubsection*{Yamabe Yang-Mills flow}

Here \(A\) is a connection on a \(U(1)\) bundle with curvature 2-form \(F\). If the \(U(1)\) bundle is trivial (vanishing first Chern class), then one may identify \(F\) with a two-form on \(M\) by precomposing with a global section of the bundle. In two dimensions, any such \(F\) satisfies \(\Tr_g F \otimes F = \tfrac{1}{2} \abs{F}^2\) where the trace contracts one slot of \(g\) with the second slot of the first \(F\) factor and the other slot of \(g\) with the second slot of the second \(F\) factor. \(A\) is required to evolve by
\[
\partial_t A = -d^{\ast} F.
\]

\subsection{The Cross Curvature Flow}
\label{subsec:flows_xcf}

\subsection{Hypersurface Flows}
\label{subsec:flows_hypersurface}

Here we have a smooth, one parameter family of immersions, \(F_t : M^n \to N^{n+1}\) satisfying,
\[
\partial_t F = f(\weingarten) \eta
\]
where \(\eta\) is a unit normal field to \(M_t = F_t(M)\), \(\weingarten\) is the Weingarten map and \(f\) is a curvature function. The metric evolves by
\[
\partial_t g = 2f(\weingarten) \sff
\]
where \(\sff\) is the second fundamental form. In other words,
\[
\speed = f(\weingarten)\sff,
\]
and
\[
\Tr_g \speed = f H, \quad \speed (\nu, \nu) = f h(\nu, \nu).
\]

\Cref{thm:general_viscosity} now becomes
\[
\begin{split}
\partial_t I &\geq I^2 I'' +  (I')^2 I + \int_{\bdry{\Omega_0}} \left(fH + R_{\Omega_0} - f\sff(\nu, \nu) - \ric(\nu)\right) \sigma \\
&\quad - \left(I'\int_{\Omega_0} fH \mu + \int_{\bdry{\Omega_0}} R_{\bdry{\Omega_0}}\sigma \right) .
\end{split}
\]

\printbibliography
\end{document}
