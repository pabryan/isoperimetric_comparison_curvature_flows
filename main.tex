\documentclass{amsart}

\input{StandardPaper2.tex}

\begin{document}

\title[Isoperimetric Comparison]
 {Isoperimetric Comparisons For Intrinsic Curvture Flows}

\curraddr{}
\email{}
\date{\today}

\dedicatory{}
\subjclass[2010]{}
\keywords{}

\begin{abstract}
\end{abstract}

\maketitle

\section{Introduction}
\label{sec:intro}

Let \(M\) be a smooth manifold and \(g = g_t\) a smooth one-parameter family of metrics evolving by,
\begin{equation}
\label{eq:dtg}
\partial_t g = 2 X.
\end{equation}

The Isoperimetric Profile is defined by,
\[
I_t (x) = I(x, t) = \inf\{\abs{\bdry{\Omega}}_{g_t} : \abs{\Omega}_{g_t} = x\}
\]
and the infimum ranges over all compactly contained, connected, open sets \(\Omega\) with smooth boundary. An isoperimetric domain is such a set \(\Omega\) satisfying \(I(\abs{\Omega}_{g_t}, t) = \abs{\bdry{\Omega_0}}_{g_t}\).

\section{A Differential Inequality For The Isoperimetric Profile}
\label{sec:iso_diff_ineq}

Let \(\Omega_0 \subset M\) be an isoperimetric domain at time \(t = t_0\) so that,
\[
I_{t_0} (\abs{\Omega_0}_{t_0}) = \abs{\bdry{\Omega_0}}_{t_0}.
\]

\printbibliography

\end{document}
